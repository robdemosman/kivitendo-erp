% ----------------------------------------------------------
%  translations.tex
%  Zentrale Uebersetzungsdatei f-tex
%
%  Changelog: see gitlog
   \newcommand{\ftTranslationsVersion}{1.2-u  (05.12.2012)}
%
%  Lizenz
%  http://www.gnu.de/licenses/gpl-3.0.html
%
%  Siehe ./README
%
%  Autor: Wulf Coulmann scripts_at_gpl.coulmann.de
%
%
% ----------------------------------------------------------


\begingroup
  \makeatletter
  \@latex@warning@no@line{ #### this is translations.tex \ftTranslationsVersion #####}
\endgroup


%%%%% Anleitung zum zufuegen neuer Sprachen %%%%%%%%%%%%%%%%%%%%%%%%%%%%%%%%%%%
% Am Beispiel Franzoesisch (fr)                                               %
% - Es wird empfohlen die Datei translations.tex im Vorlagenordner            %
%   und _nicht_ in [lxo-home]/templates/f-tex zu aendern.                     %
% - Kopiere den Block                                                         %
%     "\newcommand{\LoadDE}{"                                                 %
%   bis zur schliessenden Klammer                                             %
%     "}"                                                                     %
%   und fuege ihn am Ende der Datei bei                                       %
%     "codeblock mit neuer Sprache hier einfuegen"                            %
%   an                                                                        %
% - uebersetze die deutschen Begriffe im neu eingefuegten Block in            %
%   die neue Sprache                                                          %
% - aendere den Kommandonamen entsprechend der Neuen Sprache                  %
%     "\newcommand{\LoadFR}                                                   %
% - fuege am Ende der Datei eine neue Zeile mit dem neuen Sprachkuerzel       %
%   und dem neuen Funktionsnamen an.                                          %
%     "\IfEndWith{\docname}{_fr}{\loadFR}{}                                   %
% - pruefe, ob lxo bereits ueber eine Konfiguration zu der neuen Sprache      %
%   verfuegt. Das Feld Vorlagenkuerzel muss den zur hier zugefuegten Sprache  %
%   passenden Wert enthalten (in unserem Beispiel "fr")                       %
% - rufe das script [lxo-home]/templates/f-tex/setup.sh erneut auf, um        %
%   sicherzustellen, dass die benoetigten Symlinks vorhanden sind.            %
% - schicke die neue Version dieser Datei an                                  %
%     scripts_at_gpl.coulmann.de                                              %
%   damit in Zukunft die neue Sprache auch anderen Nutzern                    %
%   von lxo zur Verfuegung steht                                              %
%%%%%%%%%%%%%%%%%%%%%%%%%%%%%%%%%%%%%%%%%%%%%%%%%%%%%%%%%%%%%%%%%%%%%%%%%%%%%%%




% ===== de ===========
\newcommand{\loadDE}{

  \renewcommand{\TitleInv}{Rechnung}
  \newcommand{\TitleProforma}{Proformarechnung}
  \newcommand{\TitleCreditNote}{Gutschrift}
  \newcommand{\TitleSalesOrder}{Auftragsbestätigung}
  \newcommand{\TitleSalesQuotation}{Angebot}
  \newcommand{\TitleDelorder}{Lieferschein}
  \newcommand{\TitlePicklist}{Sammelliste}
  \newcommand{\TitlePurchaseOrder}{Bestellung}
  \newcommand{\DelorderNumber}{Lieferscheinnummer}
  \newcommand{\DeliveryAddress}{Lieferadresse}
  \newcommand{\InvNumber}{Rechnungsnummer}
  \newcommand{\CredNumber}{Gutschriftnummer}
  \newcommand{\OrderNumber}{Auftragsnummer}
  \newcommand{\RequestOrderNumber}{Bestellauftragsnummer}
  \newcommand{\QuotationNumber}{Angebotsnummer}
  \newcommand{\CustomerID}{Kundennummer}
  \newcommand{\VendorID}{Lieferantennummer}
  \newcommand{\DelDate}{Lieferdatum}
  \newcommand{\ReqByTitle}{Lieferung bis}
  \newcommand{\ValidUntil}{gültig bis}
  \newcommand{\Date}{Datum}
  \newcommand{\Pos}{Pos}
  \newcommand{\Number}{Best Nr.}
  \newcommand{\ItemNo}{Artikel}
  \newcommand{\Count}{Anz}
  \newcommand{\Unit}{Einh}
  \newcommand{\Storage}{Lagerplatz}
  \newcommand{\Take}{entnommen}
  \newcommand{\Fee}{Einzelp}
  \newcommand{\Total}{Total}
  \newcommand{\Sum}{Gesamtbetrag}
  \newcommand{\EbT}{Summe vor Steuern}
  \newcommand{\Left}{Restbetrag}
  \newcommand{\AlreadyPayed}{bereits gezahlt am}
  \newcommand{\TabSubTotal}{Zwischensumme}
  \newcommand{\TabCarry}{Übertrag}
  \newcommand{\Dis}{Rab}
  \newcommand{\TaxInc}{bereits enthalten: }
  \newcommand{\PriceInclTax}{Alle Preise incl. Mehrwertsteuer}
  \newcommand{\UstidTitle}{Ihre Umsatzsteueridentnummer:}


  % Zahlungshinweise
  \newcommand{\paymenthints}{

    \IfSubStr{\docname}{Angebot}{
      Das Angebot hat 4 Wochen Gültigkeit.\\
    }{}
  }

  \newcommand{\YourOrder}{
    \ifthenelse{\equal{\cusordnumber}{\leer}}
      {}
      {Ihre Bestellung {\bf\cusordnumber}}\\[0.5em]
  }

}

% ===== uk oder en ===========
\newcommand{\loadUK}{

  \renewcommand{\TitleInv}{Invoice}
  \newcommand{\TitleProforma}{Pro Forma Invoice}
  \newcommand{\TitleCreditNote}{Credit Note}
  \newcommand{\TitleSalesOrder}{Sales Order}
  \newcommand{\TitleSalesQuotation}{Sales Quotation}
  \newcommand{\DelorderNumber}{delivery note no}
  \newcommand{\DeliveryAddress}{delivery address}
  \newcommand{\TitleDelorder}{Delivery Note}
  \newcommand{\TitlePicklist}{Pick List}
  \newcommand{\TitlePurchaseOrder}{Purchase Order}
  \newcommand{\InvNumber}{invoice number}
  \newcommand{\CredNumber}{credit number}
  \newcommand{\OrderNumber}{order number}
  \newcommand{\RequestOrderNumber}{purchase order no.}
  \newcommand{\QuotationNumber}{quotation no}
  \newcommand{\CustomerID}{customer id}
  \newcommand{\VendorID}{vendor id}
  \newcommand{\DelDate}{date of delivery}
  \newcommand{\ReqByTitle}{required by}
  \newcommand{\ValidUntil}{valid until}
  \newcommand{\Date}{date}
  \newcommand{\Pos}{pos}
  \newcommand{\Number}{item id}
  \newcommand{\ItemNo}{item}
  \newcommand{\Count}{count}
  \newcommand{\Unit}{unit}
  \newcommand{\Storage}{location}
  \newcommand{\Take}{taken}
  \newcommand{\Fee}{fee}
  \newcommand{\Total}{total}
  \newcommand{\Sum}{total amount}
  \newcommand{\EbT}{total without taxes}
  \newcommand{\Left}{residue}
  \newcommand{\AlreadyPayed}{already payed at}
  \newcommand{\TabSubTotal}{subtotal}
  \newcommand{\TabCarry}{carry}
  \newcommand{\Dis}{dis}
  \newcommand{\TaxInc}{already included: }
  \newcommand{\PriceInclTax}{Prices incl. tax}
  \newcommand{\UstidTitle}{Your VAT number:}

  % Zahlungshinweise Rechnung
  \newcommand{\paymenthints}{
    \IfSubStr{\docname}{Angebot}{
      The offer is valid for 4 weeks.\\
    }{}
  }

  \newcommand{\YourOrder}{
    \ifthenelse{\equal{\cusordnumber}{\leer}}
      {}
      {Your Order Number {\bf\cusordnumber}}\\[0.5em]
  }

}

% ====== neuen Sprache ================================

   % codeblock mit neuer Sprache hier einfuegen


% ====== Ende Sprachblock =========
\newcommand{\checkVal}{unknowen}
\newcommand{\TitleInv}{\checkVal}


\IfStrEq{\LangCode}{de}{\loadDE}{}
\IfStrEq{\LangCode}{uk}{\loadUK}{}
\IfStrEq{\LangCode}{en}{\loadUK}{}
% neue Zeile mit dem neuen Sprachkuerzel und dem neuen Funktionsnamen hier anfuegen



% ====== unterhalb dieser Zeile nichts aendern ==========================

% defaultsprache
  \ifthenelse{\equal{\TitleInv}{\checkVal}}{\loadDE}{}

